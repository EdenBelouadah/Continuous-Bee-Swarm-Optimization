\chapter*{Conclusion générale}
Les machines dotées d'un système performant ont facilité le traitement et le stockage des données mais cela n'est pas suffisant parce que nous devons penser à des solutions optimales indépendantes des capacités de la machine. D'où l'intérêt de l'intelligence artificielle qui accélère le rythme des découvertes dans plusieurs domaines parmi lesquels la médecine, la chimie, l'agriculture, le commerce et l'éducation.

Les problèmes d'optimisation globale sont, depuis les deux décennies passées, un des domaines de recherche les plus compliqués .

Dans ce projet, nous avons pu montrer comment la méta-heuristique BSO, qui a été initialement conçue pour des problèmes d'optimisation combinatoire, peut être adaptée aux problèmes d'optimisation continue sans apporter beaucoup de modifications dans sa structure originale.

Les expérimentations ont montré l'efficacité de notre méta-heuristique CBSO pour les problèmes d'optimisation continue tels que \emph{Diagonal Plane, Sphere, De Joung, Easom, B2, Rosenbrock, Zakharov, Hartmann} et \emph{Branin RCOS}.

Une amélioration possible de notre algorithme serait de prendre en charge les problèmes à variables mixtes, c'est-à-dire pouvant être discrètes ou continues.

Une autre amélioration serait d'utiliser une heuristique pour restreindre les domaines de définition de la fonction problème au fur et à mesure des itérations pour augmenter le nombre de chances pour trouver l'optimum en réduisant la taille de l'espace de recherche.

Nous pourrions encore relever le défi et passer au cas de l'optimisation multi-objectif pour satisfaire plus d'un critère en même temps.

De nos jours, l'optimisation reste un domaine de recherche fécond et un sujet de plusieurs travaux de la recherche opérationnelle car elle s'applique dans tous les domaines de l'activité humaine prêtant à la modélisation mathématique.

